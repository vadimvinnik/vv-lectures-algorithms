%
% Softheme internal lectures
%
% Algorithms and data structures
%
% (c) Vadim Vinnik, 2014
%
% vadim.vinnik@gmail.com
%

\documentclass[landscape]{slides}
\usepackage[landscape, margin=10mm]{geometry}
\usepackage[russian]{babel}
\usepackage[T2A]{fontenc}
\usepackage[utf8]{inputenc}
\usepackage{amssymb}
\usepackage{amsmath}


\usepackage{textcomp}
\usepackage{hhline}

\begin{document}

\author{В.Ю.\,Винник}

\title{Алгоритмы и структуры данных\\
Часть~2: Сортировка массивов}

\date{Softheme, 30.10.2014 н.э}

\maketitle

\begin{slide}
  Классификация алгоритмов сортировки массивов
  \begin{center}
    \renewcommand{\arraystretch}{2}
    \begin{tabular}{|l|l|l|}
      \hline
      {}&
      Простой&
      Улучшенный
      \\
      \hline
      Вставка&
      Простая вставка $O(n^2)$&
      Шелла\textonesuperior{} $O(n\log^2n)$, $O(n^\frac{3}{2})$
      \\
      \hline
      Выборка&
      Простая выборка $O(n^2)$&
      Heapsort $O(n\log n)$
      \\
      \hline
      Обмен&
      Пузырьковая $O(n^2)$&
      Quicksort\texttwosuperior{} $O(n\log n)$ до $O(n^2)$
      \\
      \hline
    \end{tabular}
  \end{center}
  1) В зависимости от выбранной последовательности интервалов.\\
  2) Лучший и худший случай~-- зависит от входных данных.
\end{slide}

\begin{slide}
  Сортировка вставками
  \begin{center}
    \renewcommand{\b}[1]{\textbf{#1}}
    \begin{tabular}{l|c|c|c|c|c|c|c|c|}
      \hhline{~|=|-------}
      1&	\b{6}&	5&	3&	1&      8&      7&      2&      4\\
      \hhline{~|=|-------}
      \multicolumn{9}{r}{}\\
      \hhline{~|==|------}
      2&	\b{5}&	6&	3&	1&      8&      7&      2&      4\\
      \hhline{~|==|------}
      \multicolumn{9}{r}{}\\
      \hhline{~|===|-----}
      3&	\b{3}&	5&	6&	1&      8&      7&      2&      4\\
      \hhline{~|===|-----}
      \multicolumn{9}{r}{}\\
      \hhline{~|====|----}
      4&	\b{1}&  3&	5&	6&	8&      7&      2&      4\\
      \hhline{~|====|----}
    \end{tabular}
    \qquad
    \begin{tabular}{l|c|c|c|c|c|c|c|c|}
      \hhline{~|=====|---}
        5&	1&	3&	5&	6&	\b{8}&	7&	2&	4\\
      \hhline{~|=====|---}
      \multicolumn{9}{r}{}\\
      \hhline{~|======|--}
        6&	1&	3&	5&	6&	\b{7}&	8&	2&	4\\
      \hhline{~|======|--}
      \multicolumn{9}{r}{}\\
      \hhline{~|=======|-}
      6&	1&	\b{2}&	3&	5&	6&	7&	8&	4\\
      \hhline{~|=======|-}
      \multicolumn{9}{r}{}\\
      \hhline{~|========|}
      8&	1&	2&	3&	\b{4}&	5&	6&	7&	8\\
      \hhline{~|========|}
    \end{tabular}
  \end{center}
  Массив длиной~$n$ делится на две секции: отсортированное начало длиной~$k$ и несортированный остаток.

  Инвариант цикла: элементы $0,\ldots,k-1$ отсортированы. Начальная истинность инварианта обеспечивается при~$k=0$.

  На каждой итерации взять первый элемент несортированной секции (элемент номер~$k$) и вставить его на правильное место
  сортированной секции. Увеличить~$k$ на~1.

  Когда~$k$ достигнет~$n$, весь массив отсортирован.
\end{slide}
\end{document}

