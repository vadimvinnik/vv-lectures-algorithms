%
% Softheme internal lectures
%
% Algorithms and data structures
%
% (c) Vadim Vinnik, 2014
%
% vadim.vinnik@gmail.com
%

\documentclass[landscape]{slides}
\usepackage[landscape, margin=10mm]{geometry}
\usepackage[russian]{babel}
\usepackage[T2A]{fontenc}
\usepackage[utf8]{inputenc}
\usepackage{amssymb}
\usepackage{amsmath}



\begin{document}

\author{В.Ю.\,Винник}

\title{Алгоритмы и структуры данных\\
Часть~1: Основные определения}

\date{Softheme, 06.11.2014 н.э}

\maketitle

\begin{slide}
  Предмет этого цикла лекций:
  \begin{itemize}
      \item Алгоритмы решения типовых и важных на практике задач обработки данных;
      \item Методика создания эффективных и корректных алгоритмов: от разрозненных примеров к общим принципам;
      \item Асимптотические оценки времени работы алгоритма;
      \item Типовые структуры данных, подходящие для широкого класса практических задач и поддерживающие быстрые алгоритмы.
    \end{itemize}
\end{slide}

\begin{slide}
  Введение
  \begin{itemize}
    \item Многие современные программы не содержат нетривиальных алгоритмов~-- например, их функционирование сводится
      к формированию запросов к БД и выдаче результата с применением какого-то стиля визуального оформления.
    \item Если первые программы состояли на 99\% из вычислений и на 1\% из интерфейса, то сейчас это соотношение
      стремится к противоположному.
    \item Нельзя при этом сказать, что современные программы стали просты~-- центр сложности сместился с
      алгоритмов, заключённых внутри программы, на внешние интерфейсы к другим программам и к человеку.
    \item Однако существуют по-прежнему программы, работающие с большими объёмами данных при жёстких ограничениях на
      время отклика: поисковые системы, крупные социальные сети, САПР, системы электронного документооборота, управляющее ПО.
  \end{itemize}
\end{slide}

\begin{slide}
  Для разработки таких программ необходимо:
  \begin{itemize}
    \item Из множества известных структур данных и алгоритмов выбирать те, которые наилучшим образом отражают специфику задачи и
      обеспечивают нужные критерии качества;
    \item В случае необходимости изобретать новые структуры данных и алгоритмы.
    \item Для заданной структуры данных оценивать, какие операции над ней выполняются быстрее или медленнее по сравнению с
      другими структурами данных.
    \item Для заданного алгоритма оценивать время его работы и затраты памяти в наихудшем и среднем случае.
  \end{itemize}
\end{slide}

\begin{slide}
  Важные уточнения
  \begin{itemize}
    \item В контексте этих лекций алгоритмы и структуры данных образуют неразрывную пару.
    \item Структара данных имеет смысл только потому, что программа может выполнять над ней определённые операции,
      необходимые для решения прикладных задач;
    \item Описание алгоритма всегда включает в себя определение форматов входных и выходных данных, т.е. алгоритм всегда
      привязан к некоторой структуре данных.
  \end{itemize}
\end{slide}

\begin{slide}
  О временн\'{о}й сложности
  \begin{itemize}
    \item Нужно измерять и сравнивать время работы алгоритмов.
    \item Физическое время в секундах работы алгоритма на конкретном процессоре~-- не подходит:
      процессоры отличаются тактовой частотой, степенью распараллеливания потока команд, аппаратным ускорением
      отдельных команд или их сочетаний.
    \item Выход: измерять количество шагов машины Тьюринга (или другой абстрактной машины с хорошо определённой семантикой
      элементарных операций) или количество важных операций (сравнений, присваиваний), если алгоритм выражен на языке
      высокого уровня.
  \end{itemize}
\end{slide}

\begin{slide}
  О временн\'{о}й сложности
  \begin{itemize}
    \item Время работы алгоритма на конкретном примере входных данных~-- не показатель. Из него нельзя сделать
      вывод о предпочтительности того или иного алгоритма для применения в реальной системе, где входные данные
      варьируются в широком классе возможных значений.
    \item Выход: сравнивать \emph{асимптотику} времени работы алгоритма как функции от размера входных данных.
      Например: $f(n)=a\,n^2$ при достаточно больших~$n$ растёт быстрее, чем $g(n)=b\,n\log n$, каковы бы ни были~$a,b>0$:
      \[
        \lim_{n \to \infty} \frac{a\,n^2}{b\,n\log n} = \infty .
      \]
  \end{itemize}
\end{slide}

\begin{slide}
  Асимптотическое поведение функций. Сравнение скорости роста
  \begin{center}
    \renewcommand{\arraystretch}{2}
    \begin{tabularx}{\textwidth}{|l|l|X|}
      \hline
      $f\in\ldots$&
      Название&
      Определение
      \\
      \hline
      \hline
      $o(g)$&
      доминирование~$g$ над~$f$&
      $\exists c>0, t \forall x>t\ f(x) < c\,g(x)$
      \\
      \hline
      $O(g)$&
      ограниченность сверху&
      $\exists c>0, t \forall x>t\ f(x) \leqslant c\,g(x)$
      \\
      \hline
      $\Theta(g)$&
      ограниченность сверху и снизу&
      $f\in O(g) \land f\in \Omega(g)$
      \\
      \hline
      $\Omega(g)$&
      ограниченность снизу&
      $\exists c>0, t \forall x>t\ f(x) \geqslant c\,g(x)$
      \\
      \hline
      $\omega(g)$&
      доминирование~$f$ над~$g$&
      $\exists c>0, t \forall x>t\ f(x) > c\,g(x)$
      \\
      \hline
    \end{tabularx}
  \end{center}
  \[
    f\in\Theta(g) \Leftrightarrow \exists c_1>0,c_2>0, t\forall x>t\ c_1\, g(x) \leqslant f(x) \leqslant c_2\,g(x)
  \]
\end{slide}

\begin{slide}
  Аналогия между сравнением асимптотического поведения функций и сравнением чисел
  \begin{center}
    \renewcommand{\arraystretch}{2}
    \begin{tabular}{lcl}
      $f\in o(g)$      & $a < b$         & растёт медленнее\\
      $f\in O(g)$      & $a \leqslant b$ & растёт не быстрее\\
      $f\in \Theta(g)$ & $a = b$         & растёт столь же быстро\\
      $f\in \Omega(g)$ & $a \geqslant b$ & растёт не медленнее\\
      $f\in \omega(g)$ & $a > b$         & растёт быстрее
    \end{tabular}
  \end{center}
\end{slide}

\begin{slide}
  Основные свойства~$\Theta$
  \begin{itemize}
    \item Рефлексивность
      \[
        f\in \Theta(f) .
      \]
    \item Симметричность
      \[
        f\in \Theta(g) \Rightarrow g\in \Theta(f) .
      \]
    \item Транзитивность
      \[
        f\in \Theta(g) \land g\in \Theta(h) \Rightarrow f\in \Theta(h) .
      \]
    \item Таким образом, $\Theta$ индуцирует отношение эквивалентности на множестве функций.
  \end{itemize}
\end{slide}

\begin{slide}
  Основные свойства других классов
  \begin{itemize}
    \item Транзитивность ($\xi\in\{o,O,\Theta,\omega,\Omega\}$)
      \[
        f\in \xi(g) \land g\in \xi(h) \Rightarrow f\in \xi(h) .
      \]
    \item Перестановочная симметрия:
      \begin{eqnarray*}
        f\in O(g) &\Leftrightarrow& g\in \Omega(f),\\
        f\in o(g) &\Leftrightarrow& g\in \omega(f).
      \end{eqnarray*}
    \item Дистрибутивность ($\xi\in\{o,O,\Theta,\omega,\Omega\}$):
      \begin{eqnarray*}
        &f\in \xi(g) \Rightarrow c\,f\in \xi(g),\\
        &f, g\in\xi(h) \Rightarrow (f+g)\in\xi(h),\\
        &f\in\xi(h_1) \land g\in\xi(h_2) \Rightarrow (f\cdot g)\in\xi(h_1\cdot h_2).
      \end{eqnarray*}
    \item Поглощение
      \begin{eqnarray*}
        f\in O(h), g\in o(h) &\Rightarrow& (f+g)\in O(h),\\
        f\in O(h), g\in o(h) &\Rightarrow& (f\cdot g)\in o(h).
      \end{eqnarray*}
  \end{itemize}
\end{slide}

\begin{slide}
  Примеры
  \begin{center}
    \renewcommand{\arraystretch}{2}
    \begin{tabular}{ccc}
      $1 \in o(\log n)$&
      $\log n\in o(\sqrt{n})$&
      $\sqrt{n}\in o(n)$\\
      $n\in o(n\log n)$&
      $n\log n\in o(n^2)$&
      $n^2\in o(n^3)$\\
      $n^k\in o(p^n)$&
      $p^n\in o(n!)$&
      $n!\in O(n^n)$
    \end{tabular}
  \end{center}
\end{slide}

\begin{slide}
  Типичная временная сложность алгоритмов
  \begin{center}
    \renewcommand{\arraystretch}{1.3}
    \begin{tabularx}{\textwidth}{|l|X|}
      \hline
      Обозн.&
      Пример
      \\
      \hline
      \hline
      $O(1)$&
      Вычисление~$(-1)^n$, обращение к элементу массиву по индексу, обращение к хеш-таблице в лучшем случае
      \\
      \hline
      $O(\log n)$&
      Поиск по ключу в отсортированном массиве или сбалансированном дереве
      \\
      \hline
      $O(n)$&
      Поиск по ключу в несортированном массиве, обращение к связному списку по индексу, обращение к
      хеш-таблице в худшем случае
      \\
      \hline
      $O(n\log n)$&
      Сортировка массива слиянием, пирамидальная~-- в худшем, quicksort~-- в среднем случае. Теоретический минимум для
      сортировки сравнением. БПФ
      \\
      \hline
      $O(n^2)$&
      Сортировка массива вставками, выборками, пузырьковая. Quicksort в худшем случае.
      \\
      \hline
      $O(c^n)$&
      Распознавание эквивалентности формул логики высказываний полным перебором. Решение задачи коммивояжёра
      методом динамического программирования.
      \\
      \hline
      $O(n!)$&
      Решение задачи коммивояжёра полным перебором.
      \\
      \hline
    \end{tabularx}
  \end{center}
\end{slide}
\end{document}

