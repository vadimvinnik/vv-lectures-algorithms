%
% Softheme internal lectures
%
% Algorithms and data structures
%
% (c) Vadim Vinnik, 2014
%
% vadim.vinnik@gmail.com
%

\documentclass[landscape]{slides}
\usepackage[landscape, margin=10mm]{geometry}
\usepackage[russian]{babel}
\usepackage[T2A]{fontenc}
\usepackage[utf8]{inputenc}
\usepackage{amssymb}
\usepackage{amsmath}



\begin{document}

\author{В.Ю.\,Винник}

\title{Алгоритмы и структуры данных\\
Часть~1}

\date{Softheme, 30.10.2014 н.э}

\maketitle

\begin{slide}
  Предмет этого цикла лекций:
  \begin{itemize}
      \item Алгоритмы решения типовых и важных на практике задач обработки данных;
      \item Методика создания эффективных и корректных алгоритмов: от разрозненных примеров к общим принципам;
      \item Оценки времени работы алгоритма, асимптотические оценки: $O(n)$, $\Omega(n)$, $\Theta(n)$ и др.;
      \item Типовые структуры данных, подходящие для широкого класса практических задач и поддерживающие быстрые алгоритмы.
    \end{itemize}
\end{slide}

\begin{slide}
  Введение
  \begin{itemize}
    \item Многие современные программы не содержат нетривиальных алгоритмов~-- например, их функционирование сводится
      к формированию запросов к БД и выдаче результата с применением какого-то стиля.
    \item Если первые программы состояли на 99\% из вычислений и на 1\% из красивого интерфейса, то сейчас это соотношение
      имеет тенденцию к переворачиванию.
    \item Нельзя при этом сказать, что современные программы стали просты~-- центр сложности сместился с
      алгоритмов, заключённых внутри программы, на интерфейсы во внешний мир: к другим программам и к человеку.
    \item Таким образом, данная лекция совершенно бесполезна, а изучение этого материала есть пустая трата времени,
      если слушатель планирует создавать небольшие веб-магазины, небольшие платформы для блогов или форумов, скрипты

\end{document}

